\documentclass[a4paper, 12pt]{extarticle}
\usepackage[utf8]{inputenc}
\usepackage[russian]{babel}
\usepackage{longtable, moreverb}
\usepackage{ amssymb, latexsym, amsmath, amsthm}
\newtheorem{myth}{Теорема}
\usepackage{verbatim}
\textwidth=17cm
\textheight=20cm
\oddsidemargin=-0.7cm

\sloppy

\xdef\LastDeclaredEncoding{T2A}
% Переносы математики.
\begingroup
\catcode`\+\active\gdef+{\mathchar8235\nobreak\discretionary{}{\usefont{OT1}{cmr}{m}{n}\char43}{}}
\catcode`\-\active\gdef-{\mathchar8704\nobreak\discretionary{}{\usefont{OMS}{cmsy}{m}{n}\char0}{}}
\catcode`\=\active\gdef={\mathchar12349\nobreak\discretionary{}{\usefont{OT1}{cmr}{m}{n}\char61}{}}
\catcode`\<\active\gdef<{\mathchar"313C\nobreak\discretionary{}{\usefont{OML}{cmm}{m}{n}\char60}{}}
\catcode`\>\active\gdef>{\mathchar"313E\nobreak\discretionary{}{\usefont{OML}{cmm}{m}{n}\char62}{}}
\endgroup
\def\times{\mathchar8706\nobreak\discretionary{}{\usefont{OMS}{cmsy}{m}{n}\char2}{}}
\def\subset{\mathchar"321A\nobreak\discretionary{}{\usefont{OMS}{cmsy}{m}{n}\char26}{}}
%\supset,\subseteq,\notin
\def\supset{\mathchar"321B\nobreak\discretionary{}{\usefont{OMS}{cmsy}{m}{n}\char27}{}}
\def\subseteq{\mathchar"3212\nobreak\discretionary{}{\usefont{OMS}{cmsy}{m}{n}\char18}{}}
\def\neq{\not=\nobreak\discretionary{}{\usefont{OMS}{cmsy}{m}{n}\char54\usefont{OT1}{cmr}{m}{n}\char61}{}}
\def\sim{\mathchar"3218\nobreak\discretionary{}{\usefont{OMS}{cmsy}{m}{n}\char24}{}}
\def\in{\mathchar"3232\nobreak\discretionary{}{\usefont{OMS}{cmsy}{m}{n}\char50}{}}
\def\to{\mathchar"3221\nobreak\discretionary{}{\usefont{OMS}{cmsy}{m}{n}\char33}{}}
\def\?#1{#1\nobreak\discretionary{}{\hbox{$\mathsurround=0pt #1$}}{}}
% Конец переносов математики.

\begin{document}
% Восстанавливаем после amsmath.
\begingroup \catcode`\"=12
\gdef\newmcodes@{\mathcode`\'39\mathcode`\*42\mathcode`\."613A%
\mathcode`\-"8000\mathcode`\/47\mathcode`\:"603A\relax}%
\endgroup
\mathcode`\=="8000 \mathcode`\+="8000 \mathcode`\-="8000
\mathcode`\<="8000 \mathcode`\>="8000

%\tableofcontents

%\newpage

\setcounter{secnumdepth}{-1}

\section{Введение}
Одним из стандартных способов задания функций k\nobreakdash-значной логики являются поляризованные полиномиальные формы (ППФ),
которые также называются обобщенными формами Рида-Мюллера, или каноническими поляризованными полиномами. В ППФ каждая переменная
имеет определенную поляризацию. Длиной полиномиальной формы называется число попарно различных слагаемых в ней. Длиной функции
$F$ в классе ППФ называется наименьшая длина среди длин всех поляризованных полиномиальных форм, реализующих $F$.
%Для функции k\nobreakdash-значной логики $F$ сложностью в классе поляризованных полиномиальных форм называется
%длина кратчайшей ППФ, реализующей~$F$.
%~ Поляризованные полиномиальные формы не 
Функция Шеннона $L^K_k(n)$ длины определяется как наибольшая длина среди всех функций $k$\nobreakdash-значной логики в классе $K$
от~$n$~переменных, если $K$ опущено, то подразумевается класс ППФ.
Практическое применение ППФ нашли при построении программируемых логических матриц (ПЛМ)~\cite{ue04, sb90}, сложность ПЛМ
напрямую зависит от длины ППФ, по которой она построена. Поэтому в ряде работ исследуется сложность ППФ различных функций.

В 1993  В.\,П.\,Супрун~\cite{sv93} получил первые оценки функции Шеннона для функций алгебры логики :
$$
L_2(n) \geqslant C_n^{[\frac{n}{2}]},
$$
$$
L_2(n) < 3 \cdot 2^{n-1}.
$$
где [$a$] обозначает целую часть $a$.

Точное значение функции Шеннона для функций алгебры логики в 1995\,г. было
найдено Н.\,А.\,Перязевым~\cite{pn95} :
$$
L_2(n) = \left[\frac{2^{n+1}}{3}\right].
$$

Функции $k$\nobreakdash-значных логик являются естественным обобщением функций алгебры логики.
Для функций $k$\nobreakdash-значной логики верхняя оценка функции Шеннона была получена в 2002\,г. С.\,Н.\,Селезневой~\cite{ss02} :
$$
L_k(n) < \frac{k(k-1)}{k(k-1)+1}k^n.
$$

При построении ПЛМ рассматривают и другие полиномиальные формы. Например класс обобщенных полиномиальных форм.
В классе обобщенных полиномиальных форм, в отличие от класса поляризованных полиномиальных форм, переменные могут иметь
различную поляризацию в разных слагаемых. В статье К.\,Д.\,Кириченко~\cite{kk05}, опубликованной в 2005\,г., получена верхняя оценка
функции Шеннона в классе обобщенных полиномиальных форм функций алгебры логики :
$$
L^{\text{О.П.}}_2(n) < \frac{2 ^ {n + 1}(\log_2n+1)}{n}.
$$

Верхняя оценка функции Шеннона в классе обобщенных полиномиальных форм функций k\nobreakdash-значной логики была получена
С.\,Н.\,Селезневой А.\,Б.\,Дайняком в 2008\,г.~\cite{sd08}:
$$
L^{\text{О.П.}}_k(n) \lesssim 2\cdot\frac{k ^ n}{n}\cdot \ln n \text{ при } n \rightarrow \infty.
$$

В 2012\,г. Н.\,К.\,Маркеловым была получена нижняя оценка функции Шеннона для функции трехзначной логики в классе
поляризованных полиномов~\cite{mn12}:
$$
L_3(n) \geqslant \left[\frac{3}{4}3^n\right].
$$

\section{Основные определения}

Пусть $k \geqslant 2$ -- натуральное число,
$E_k = \{0, 1, \dots, k - 1\}$
. Весом набора
$\alpha = (a_1, \dots, a_n ) \in E_k^n$ назовем число $|\alpha| = \sum\limits_{i=1}^n a_i$.
Моном $\prod\limits_{a_i\neq0}x_i^{a_i}$ назовем соответствующим набору $\alpha = (a_1, \dots, a_n ) \in E_k^n$ и обозначим
через $K_{\alpha}$. По определению положим, что константа 1 соответствует набору из всех нулей.
Функцией $k$\nobreakdash-значной логики называется отображение $f^{(n)} : E_k^n \rightarrow E_k$,
$n = 0, 1, \dots$ . Множество всех $k$\nobreakdash-значных функций обозначим через $P_k$ , множество
всех $k$\nobreakdash-значных функций, зависящих от переменных $x_1, \dots, x_n$ , обозначим через $P_k^n$ .

Если $k$ -- простое число, то каждая функция $k$\nobreakdash-значной логики $f(x_1 , \dots , x_n)$
может быть однозначно задана формулой вида

$$ f(x_1, \dots, x_n) = \sum_{\alpha \in E_k^n:c_f(\alpha) \neq 0}c_f(\alpha)K_\alpha \; ,$$
где $c_f(\alpha) \in E_k$ -- коэффициенты, $\alpha \in E_k$, и операции сложения и умножения
рассматриваются по модулю $k$. Это представление функций $k$\nobreakdash-значной
логики называется ее полиномом по модулю $k$. При простых $k$ однозначно
определенный полином по модулю k для функции $k$\nobreakdash-значной логики $f$ будем
обозначать через $P(f)$.

Определим поляризованные полиномиальные формы по модулю $k$. Поляризованной переменной $x_i$ с поляризацией $d$,
$d \in E_k$ , назовем выражение вида $(x_i + d)$. Поляризованным мономом по вектору поляризации $\delta$,
$\delta = (d_1, \dots, d_n) \in E_k^n$, назовем произведение вида $(x_{i_1} + d_{i_1} )^{m_1}\cdots(x_{i_r} + d_{i_r})^{m_r}$,
где $1 \leqslant i_1 < \ldots < i_r \leqslant n$, и $1 \leqslant m 1 , \dots , m_r \leqslant k - 1$. Обычный моном является
мономом, поляризованным по вектору $\tilde{0} = (0, \dots, 0) \in E_k^n $

Выражение вида $\sum\limits_{i=1}^lc_i \cdot K_i$, где $c_i \in E_k\setminus\{0\}$ -- коэффициенты, $K_i$ -- попарно
различные мономы, поляризованные по вектору $\delta = (d_1, \dots, d_n) \in E_k^n$, $i = 1, \dots , l$, назовем
поляризованной полиномиальной нормальной формой (ППФ) по вектору поляризации $\delta$. Мы будем считать, что константа 0
является ППФ по произвольному вектору поляризации. Заметим, что при простых $k$ для каждого вектора поляризации каждую функцию
$k$\nobreakdash-значной логики можно однозначно представить ППФ по этому вектору поляризации \cite{ss02}. При простых $k$
однозначно определенную ППФ по вектору поляризации $\delta \in E_k^n$ для функции
$f \in P_k^n$ будем обозначать через $P^{\delta}(f)$.

Длиной $l(p)$ ППФ $p$ назовем число попарно различных слагаемых в этой
ППФ. Положим, что $l(0) = 0$. При простых $k$ длиной функции $k$\nobreakdash-значной
логики в классе ППФ называется величина $l^{\text{ППФ}}(f) = \min\limits_{\delta \in E_k^n}l(P^{\delta}(f))$.

Сложностью системы ППФ, имеющих один и тот же вектор поляризации, называется число попарно различных слагаемых,
встречающихся во всех этих ППФ. При простых $k$ сложностью $L_k^{\text{ППФ}}(F)$ системы функций $k$\nobreakdash-значной
логики $F = \{f_1(x_1 , \dots , x_n ), \dots , f_m (x_1 , \dots , x_n )\}$ в классе ППФ называется минимальная сложность
среди всех таких систем ППФ $\{p_1 , \dots , p_m \}$, что все ППФ $p_1 , \dots , p_m$ имеют один и тот же вектор поляризации,
и ППФ $p_j$ реализует функцию $f_j,\; j = 1, \dots , m$. Понятно, что для произвольной системы функций $k$\nobreakdash-значной
логики $F = \{f_1(x_1, \dots, x_n), \dots , f_m(x_1 , \dots , x_n )\}$ верна оценка $L_k^{\text{ППФ}}(F) \leqslant k^n$.

Пусть $k$ -- простое число, и $A_k \subseteq P_k$ , а $A^n_k = A_k \cap P_k^n$ . Введем функцию
Шеннона $L^{\text{ППФ}}_{A_k}(m, n)$ сложности систем функций $k$\nobreakdash-значной логики, принадлежащих множеству $A$,
в классе ППФ:
$$L^{\text{ППФ}}_{A_k}(m, n) = \max_{B\subseteq A_k^n, |B|=m}L^{\text{ППФ}}_{k}(B) .$$

Если $A_k = P_k$ , то функцию Шеннона будем обозначать через $L^{\text{ППФ}}_{A_k}(m, n)$.

Функция $k$\nobreakdash-значной логики $f(x_1 ,\dots , x_n)$ называется симметрической, если
$$f(\pi(x_1), \dots, \pi(x_n)) = f(x_1, \dots, x_n)$$
для произвольной перестановки $\pi$ на множестве переменных $\{x_1 , \dots , x_n \}$.
Множество всех симметрических функций $k$\nobreakdash-значной логики обозначим через $S_k$.
Симметрическая функция $f(x_1, \dots, x_n)$ называется периодической c
периодом $\tau = (\tau_0 \tau_1 \dots \tau_{T-1}) \in E_k^T$ , если $f(\alpha) = \tau_j$ при $|\alpha| = j(mod T)$
для каждого набора $\alpha \in E_k^n$. При этом число $T$ называется длиной периода. Периодическую функцию
$k$\nobreakdash-значной логики $f(x_1 , \dots , x_n)$ с периодом $\tau = (\tau_0 \tau_1 \dots \tau_{T-1}) \in E_k^T$
будем обозначать через $f^{(n)}_{(\tau_0 \tau_1 \dots \tau_{T-1})}$. Понятно, что
такое обозначение полностью определяет эту функцию.

Введем функцию $rol(\alpha, i) \in E_k^n \times E_k \rightarrow E_k^n$, производящую чиклический сдвиг вектора $\alpha$
влево. Пусть $\alpha = (a_1, \dots, a_n)$, тогда $rol(\alpha, i) = (a_{(1+i)\mod k}, \dots, a_{(n+i)\mod k})$.

\section{Результаты}

\begin{comment}
\begin{myth} При $n \geqslant 1 $ для периодических функций пятизначной логики $f_n = f^{\left(n\right)}_{\left(1144\right)}$,
$g_n = f^{\left(n\right)}_{\left(1441\right)}$ верны следующие равенства:
\end{myth}

$ f_{n+1} = j_0(x_{n+1})f_n + j_1(x_{n+1})g_n - j_2(x_{n+1})f_n - j_3(x_{n+1})g_n + j_4(x_{n+1})f_n
 = 4 \, f_{n} x_{n+1}^{4} + {\left(3 \, f_{n} + 2 \, g_{n}\right)} x_{n+1}^{3} + {\left(3 \, f_{n} + 3 \, g_{n}\right)} x_{n+1}^{2} + {\left(4 \, f_{n} + g_{n}\right)} x_{n+1} + f_{n}
 = 4 \, f_{n} {\left(x_{n+1} + 1\right)}^{4} + {\left(2 \, f_{n} + 2 \, g_{n}\right)} {\left(x_{n+1} + 1\right)}^{3} + {\left(3 \, f_{n} + 2 \, g_{n}\right)} {\left(x_{n+1} + 1\right)}^{2} + {\left(f_{n} + g_{n}\right)} {\left(x_{n+1} + 1\right)} + f_{n}
 = 4 \, f_{n} {\left(x_{n+1} + 2\right)}^{4} + {\left(f_{n} + 2 \, g_{n}\right)} {\left(x_{n+1} + 2\right)}^{3} + {\left(f_{n} + g_{n}\right)} {\left(x_{n+1} + 2\right)}^{2} + 3 \, g_{n} {\left(x_{n+1} + 2\right)} + 4 \, g_{n}
 = 4 \, f_{n} {\left(x_{n+1} + 3\right)}^{4} + 2 \, g_{n} {\left(x_{n+1} + 3\right)}^{3} + 2 \, f_{n} {\left(x_{n+1} + 3\right)}^{2} + 2 \, g_{n} {\left(x_{n+1} + 3\right)} + 4 \, f_{n}
 = 4 \, f_{n} {\left(x_{n+1} + 4\right)}^{4} + {\left(4 \, f_{n} + 2 \, g_{n}\right)} {\left(x_{n+1} + 4\right)}^{3} + {\left(f_{n} - g_{n}\right)} {\left(x_{n+1} + 4\right)}^{2} + 3 \, g_{n} {\left(x_{n+1} + 4\right)} + g_{n}
$

\begin{myth} При $n \geqslant 1 $ для периодических функций пятизначной логики $f_n = f^{\left(n\right)}_{\left(1144\right)}$,
$g_n = f^{\left(n\right)}_{\left(1441\right)}$ верны следующие равенства:
\end{myth}

$ g_{n+1} = j_0(x_{n+1})g_n - j_1(x_{n+1})f_n - j_2(x_{n+1})g_n + j_3(x_{n+1})f_n + j_4(x_{n+1})g_n
 = 4 \, g_{n} x_{n+1}^{4} + {\left(3 \, f_{n} + 3 \, g_{n}\right)} x_{n+1}^{3} + {\left(2 \, f_{n} + 3 \, g_{n}\right)} x_{n+1}^{2} + {\left(4 \, f_{n} + 4 \, g_{n}\right)} x_{n+1} + g_{n}
 = 4 \, g_{n} {\left(x_{n+1} + 1\right)}^{4} + {\left(3 \, f_{n} + 2 \, g_{n}\right)} {\left(x_{n+1} + 1\right)}^{3} + {\left(3 \, f_{n} + 3 \, g_{n}\right)} {\left(x_{n+1} + 1\right)}^{2} + {\left(4 \, f_{n} + g_{n}\right)} {\left(x_{n+1} + 1\right)} + g_{n}
 = 4 \, g_{n} {\left(x_{n+1} + 2\right)}^{4} + {\left(3 \, f_{n} + g_{n}\right)} {\left(x_{n+1} + 2\right)}^{3} + 4 \, {\left(f_{n} + 4 \, g_{n}\right)} {\left(x_{n+1} + 2\right)}^{2} + 2 \, f_{n} {\left(x_{n+1} + 2\right)} + f_{n}
 = 4 \, g_{n} {\left(x_{n+1} + 3\right)}^{4} + 3 \, f_{n} {\left(x_{n+1} + 3\right)}^{3} + 2 \, g_{n} {\left(x_{n+1} + 3\right)}^{2} + 3 \, f_{n} {\left(x_{n+1} + 3\right)} + 4 \, g_{n}
 = 4 \, g_{n} {\left(x_{n+1} + 4\right)}^{4} + {\left(3 \, f_{n} + 4 \, g_{n}\right)} {\left(x_{n+1} + 4\right)}^{3} + {\left(f_{n} + g_{n}\right)} {\left(x_{n+1} + 4\right)}^{2} + 2 \, f_{n} {\left(x_{n+1} + 4\right)} + 4 \, f_{n}
$
\end{comment}

\begin{myth} При $n \geqslant 1 $ для периодических функций пятизначной логики $f_n = f^{\left(n\right)}_{\left(1144\right)}$,
$g_n = f^{\left(n\right)}_{\left(1441\right)}$ верны следующие равенства:
$$\begin{array}{l}
 f_{n+1} = j_0(x_{n+1})f_n + j_1(x_{n+1})g_n + 4\,j_2(x_{n+1})f_n + 4\,j_3(x_{n+1})g_n + j_4(x_{n+1})f_n =\\
4 \, f_{n} x_{n+1}^{4} + {\left(3 \, f_{n} + 2 \, g_{n}\right)} x_{n+1}^{3} + 3 \, {\left(f_{n} + g_{n}\right)} x_{n+1}^{2} + {\left(4 \, f_{n} + g_{n}\right)} x_{n+1} + f_{n}=\\
4 \, f_{n} {\left(x_{n+1} + 1\right)}^{4} + 2 \, {\left(f_{n} + g_{n}\right)} {\left(x_{n+1} + 1\right)}^{3} + {\left(3 \, f_{n} + 2 \, g_{n}\right)} {\left(x_{n+1} + 1\right)}^{2} + {\left(f_{n} + g_{n}\right)} {\left(x_{n+1} + 1\right)} + f_{n}=\\
4 \, f_{n} {\left(x_{n+1} + 2\right)}^{4} + {\left(f_{n} + 2 \, g_{n}\right)} {\left(x_{n+1} + 2\right)}^{3} + {\left(f_{n} + g_{n}\right)} {\left(x_{n+1} + 2\right)}^{2} + 3 \, g_{n} {\left(x_{n+1} + 2\right)} + 4 \, g_{n}=\\
4 \, f_{n} {\left(x_{n+1} + 3\right)}^{4} + 2 \, g_{n} {\left(x_{n+1} + 3\right)}^{3} + 2 \, f_{n} {\left(x_{n+1} + 3\right)}^{2} + 2 \, g_{n} {\left(x_{n+1} + 3\right)} + 4 \, f_{n}=\\
4 \, f_{n} {\left(x_{n+1} + 4\right)}^{4} + 2 \, {\left(2 \, f_{n} + g_{n}\right)} {\left(x_{n+1} + 4\right)}^{3} + {\left(f_{n} + 4 \, g_{n}\right)} {\left(x_{n+1} + 4\right)}^{2} + 3 \, g_{n} {\left(x_{n+1} + 4\right)} + g_{n}=\\
\end{array}$$
\end{myth}

\begin{proof}

Первое равенство следует из теоремы 1.

При поляризации $x_{n+1}$, когда $d_{n+1} = 0$
$$\begin{array}{l}
f_{n+1}(\bar{x}_n, 0) = 0 + 0 + 0 + 0 + f_n = f_n \\
f_{n+1}(\bar{x}_n, 1) = 4\,f_n + 3\,f_n + 2\,g_n + 3\,f_n + 3\,g_n + 4\,f_n + g_n + f_n = g_n \\
f_{n+1}(\bar{x}_n, 2) = 4\,f_n + 4\,f_n + g_n + 2\,f_n + 2\,g_n + 3\,f_n + 2\,g_n + f_n = 4\,f_n \\
f_{n+1}(\bar{x}_n, 3) = 4\,f_n + f_n + 4\,g_n + 2\,f_n + 2\,g_n + 2\,f_n + 3\,g_n + f_n = 4\,g_n \\
f_{n+1}(\bar{x}_n, 4) = 4\,f_n + 2\,f_n + 3\,g_n + 3\,f_n + 3\,g_n + f_n + 4\,g_n + f_n = f_n \\
\end{array}$$
При поляризации $x_{n+1}$, когда $d_{n+1} = 1$
$$\begin{array}{l}
f_{n+1}(\bar{x}_n, 0) = 4\,f_n + 2\,f_n + 2\,g_n + 3\,f_n + 2\,g_n + f_n + g_n + f_n = f_n \\
f_{n+1}(\bar{x}_n, 1) = 4\,f_n + f_n + g_n + 2\,f_n + 3\,g_n + 2\,f_n + 2\,g_n + f_n = g_n \\
f_{n+1}(\bar{x}_n, 2) = 4\,f_n + 4\,f_n + 4\,g_n + 2\,f_n + 3\,g_n + 3\,f_n + 3\,g_n + f_n = 4\,f_n \\
f_{n+1}(\bar{x}_n, 3) = 4\,f_n + 3\,f_n + 3\,g_n + 3\,f_n + 2\,g_n + 4\,f_n + 4\,g_n + f_n = 4\,g_n \\
f_{n+1}(\bar{x}_n, 4) = 0 + 0 + 0 + 0 + f_n = f_n \\
\end{array}$$
При поляризации $x_{n+1}$, когда $d_{n+1} = 2$
$$\begin{array}{l}
f_{n+1}(\bar{x}_n, 0) = 4\,f_n + 3\,f_n + g_n + 4\,f_n + 4\,g_n + g_n + 4\,g_n = f_n \\
f_{n+1}(\bar{x}_n, 1) = 4\,f_n + 2\,f_n + 4\,g_n + 4\,f_n + 4\,g_n + 4\,g_n + 4\,g_n = g_n \\
f_{n+1}(\bar{x}_n, 2) = 4\,f_n + 4\,f_n + 3\,g_n + f_n + g_n + 2\,g_n + 4\,g_n = 4\,f_n \\
f_{n+1}(\bar{x}_n, 3) = 0 + 0 + 0 + 0 + 4\,g_n = 4\,g_n \\
f_{n+1}(\bar{x}_n, 4) = 4\,f_n + f_n + 2\,g_n + f_n + g_n + 3\,g_n + 4\,g_n = f_n \\
\end{array}$$
При поляризации $x_{n+1}$, когда $d_{n+1} = 3$
$$\begin{array}{l}
f_{n+1}(\bar{x}_n, 0) = 4\,f_n + 4\,g_n + 3\,f_n + g_n + 4\,f_n = f_n \\
f_{n+1}(\bar{x}_n, 1) = 4\,f_n + 3\,g_n + 2\,f_n + 3\,g_n + 4\,f_n = g_n \\
f_{n+1}(\bar{x}_n, 2) = 0 + 0 + 0 + 0 + 4\,f_n = 4\,f_n \\
f_{n+1}(\bar{x}_n, 3) = 4\,f_n + 2\,g_n + 2\,f_n + 2\,g_n + 4\,f_n = 4\,g_n \\
f_{n+1}(\bar{x}_n, 4) = 4\,f_n + g_n + 3\,f_n + 4\,g_n + 4\,f_n = f_n \\
\end{array}$$
При поляризации $x_{n+1}$, когда $d_{n+1} = 4$
$$\begin{array}{l}
f_{n+1}(\bar{x}_n, 0) = 4\,f_n + f_n + 3\,g_n + f_n + 4\,g_n + 2\,g_n + g_n = f_n \\
f_{n+1}(\bar{x}_n, 1) = 0 + 0 + 0 + 0 + g_n = g_n \\
f_{n+1}(\bar{x}_n, 2) = 4\,f_n + 4\,f_n + 2\,g_n + f_n + g_n + 3\,g_n + g_n = 4\,f_n \\
f_{n+1}(\bar{x}_n, 3) = 4\,f_n + 2\,f_n + g_n + 4\,f_n + g_n + g_n + g_n = 4\,g_n \\
f_{n+1}(\bar{x}_n, 4) = 4\,f_n + 3\,f_n + 4\,g_n + 4\,f_n + g_n + 4\,g_n + g_n = f_n \\
\end{array}$$

\end{proof}

\begin{myth} При $n \geqslant 1 $ для периодических функций пятизначной логики $f_n = f^{\left(n\right)}_{\left(1144\right)}$,
$g_n = f^{\left(n\right)}_{\left(1441\right)}$ верны следующие равенства:
$$\begin{array}{l}
 g_{n+1} = j_0(x_{n+1})g_n + 4\,j_1(x_{n+1})f_n + 4\,j_2(x_{n+1})g_n + j_3(x_{n+1})f_n + j_4(x_{n+1})g_n=\\
4 \, g_{n} x_{n+1}^{4} + 3 \, {\left(f_{n} + g_{n}\right)} x_{n+1}^{3} + {\left(2 \, f_{n} + 3 \, g_{n}\right)} x_{n+1}^{2} + 4 \, {\left(f_{n} + g_{n}\right)} x_{n+1} + g_{n}=\\
4 \, g_{n} {\left(x_{n+1} + 1\right)}^{4} + {\left(3 \, f_{n} + 2 \, g_{n}\right)} {\left(x_{n+1} + 1\right)}^{3} + 3 \, {\left(f_{n} + g_{n}\right)} {\left(x_{n+1} + 1\right)}^{2} + {\left(4 \, f_{n} + g_{n}\right)} {\left(x_{n+1} + 1\right)} + g_{n}=\\
4 \, g_{n} {\left(x_{n+1} + 2\right)}^{4} + {\left(3 \, f_{n} + g_{n}\right)} {\left(x_{n+1} + 2\right)}^{3} + {\left(4 \, f_{n} + g_{n}\right)} {\left(x_{n+1} + 2\right)}^{2} + 2 \, f_{n} {\left(x_{n+1} + 2\right)} + f_{n}=\\
4 \, g_{n} {\left(x_{n+1} + 3\right)}^{4} + 3 \, f_{n} {\left(x_{n+1} + 3\right)}^{3} + 2 \, g_{n} {\left(x_{n+1} + 3\right)}^{2} + 3 \, f_{n} {\left(x_{n+1} + 3\right)} + 4 \, g_{n}=\\
4 \, g_{n} {\left(x_{n+1} + 4\right)}^{4} + {\left(3 \, f_{n} + 4 \, g_{n}\right)} {\left(x_{n+1} + 4\right)}^{3} + {\left(f_{n} + g_{n}\right)} {\left(x_{n+1} + 4\right)}^{2} + 2 \, f_{n} {\left(x_{n+1} + 4\right)} + 4 \, f_{n}=\\
\end{array}$$
\end{myth}

\begin{proof}

Первое равенство следует из теоремы 1.

При поляризации $x_{n+1}$, когда $d_{n+1} = 0$
$$\begin{array}{l}
g_{n+1}(\bar{x}_n, 0) = 0 + 0 + 0 + 0 + g_n = g_n \\
g_{n+1}(\bar{x}_n, 1) = 4\,g_n + 3\,f_n + 3\,g_n + 2\,f_n + 3\,g_n + 4\,f_n + 4\,g_n + g_n = 4\,f_n \\
g_{n+1}(\bar{x}_n, 2) = 4\,g_n + 4\,f_n + 4\,g_n + 3\,f_n + 2\,g_n + 3\,f_n + 3\,g_n + g_n = 4\,g_n \\
g_{n+1}(\bar{x}_n, 3) = 4\,g_n + f_n + g_n + 3\,f_n + 2\,g_n + 2\,f_n + 2\,g_n + g_n = f_n \\
g_{n+1}(\bar{x}_n, 4) = 4\,g_n + 2\,f_n + 2\,g_n + 2\,f_n + 3\,g_n + f_n + g_n + g_n = g_n \\
\end{array}$$
При поляризации $x_{n+1}$, когда $d_{n+1} = 1$
$$\begin{array}{l}
g_{n+1}(\bar{x}_n, 0) = 4\,g_n + 3\,f_n + 2\,g_n + 3\,f_n + 3\,g_n + 4\,f_n + g_n + g_n = g_n \\
g_{n+1}(\bar{x}_n, 1) = 4\,g_n + 4\,f_n + g_n + 2\,f_n + 2\,g_n + 3\,f_n + 2\,g_n + g_n = 4\,f_n \\
g_{n+1}(\bar{x}_n, 2) = 4\,g_n + f_n + 4\,g_n + 2\,f_n + 2\,g_n + 2\,f_n + 3\,g_n + g_n = 4\,g_n \\
g_{n+1}(\bar{x}_n, 3) = 4\,g_n + 2\,f_n + 3\,g_n + 3\,f_n + 3\,g_n + f_n + 4\,g_n + g_n = f_n \\
g_{n+1}(\bar{x}_n, 4) = 0 + 0 + 0 + 0 + g_n = g_n \\
\end{array}$$
При поляризации $x_{n+1}$, когда $d_{n+1} = 2$
$$\begin{array}{l}
g_{n+1}(\bar{x}_n, 0) = 4\,g_n + 4\,f_n + 3\,g_n + f_n + 4\,g_n + 4\,f_n + f_n = g_n \\
g_{n+1}(\bar{x}_n, 1) = 4\,g_n + f_n + 2\,g_n + f_n + 4\,g_n + f_n + f_n = 4\,f_n \\
g_{n+1}(\bar{x}_n, 2) = 4\,g_n + 2\,f_n + 4\,g_n + 4\,f_n + g_n + 3\,f_n + f_n = 4\,g_n \\
g_{n+1}(\bar{x}_n, 3) = 0 + 0 + 0 + 0 + f_n = f_n \\
g_{n+1}(\bar{x}_n, 4) = 4\,g_n + 3\,f_n + g_n + 4\,f_n + g_n + 2\,f_n + f_n = g_n \\
\end{array}$$
При поляризации $x_{n+1}$, когда $d_{n+1} = 3$
$$\begin{array}{l}
g_{n+1}(\bar{x}_n, 0) = 4\,g_n + f_n + 3\,g_n + 4\,f_n + 4\,g_n = g_n \\
g_{n+1}(\bar{x}_n, 1) = 4\,g_n + 2\,f_n + 2\,g_n + 2\,f_n + 4\,g_n = 4\,f_n \\
g_{n+1}(\bar{x}_n, 2) = 0 + 0 + 0 + 0 + 4\,g_n = 4\,g_n \\
g_{n+1}(\bar{x}_n, 3) = 4\,g_n + 3\,f_n + 2\,g_n + 3\,f_n + 4\,g_n = f_n \\
g_{n+1}(\bar{x}_n, 4) = 4\,g_n + 4\,f_n + 3\,g_n + f_n + 4\,g_n = g_n \\
\end{array}$$
При поляризации $x_{n+1}$, когда $d_{n+1} = 4$
$$\begin{array}{l}
g_{n+1}(\bar{x}_n, 0) = 4\,g_n + 2\,f_n + g_n + f_n + g_n + 3\,f_n + 4\,f_n = g_n \\
g_{n+1}(\bar{x}_n, 1) = 0 + 0 + 0 + 0 + 4\,f_n = 4\,f_n \\
g_{n+1}(\bar{x}_n, 2) = 4\,g_n + 3\,f_n + 4\,g_n + f_n + g_n + 2\,f_n + 4\,f_n = 4\,g_n \\
g_{n+1}(\bar{x}_n, 3) = 4\,g_n + 4\,f_n + 2\,g_n + 4\,f_n + 4\,g_n + 4\,f_n + 4\,f_n = f_n \\
g_{n+1}(\bar{x}_n, 4) = 4\,g_n + f_n + 3\,g_n + 4\,f_n + 4\,g_n + f_n + 4\,f_n = g_n \\
\end{array}$$

\end{proof}

\begin{myth} При $n \geqslant 1 $ для периодических функций пятизначной логики $f_n = f^{\left(n\right)}_{\left(1144\right)}$,
$g_n = f^{\left(n\right)}_{\left(1441\right)}$ верны следующие равенства:
$$\begin{array}{l}
s_{n+1}^1 = f_{n+1} + g_{n+1}=\\
 {\left(4 \, f_{n} + 4 \, g_{n}\right)} x^{4} + f_{n} x^{3} + g_{n} x^{2} + 3 \, f_{n} x + f_{n} + g_{n} =\\
 {\left(4 \, f_{n} + 4 \, g_{n}\right)} {\left(x + 1\right)}^{4} + 4 \, g_{n} {\left(x + 1\right)}^{3} + f_{n} {\left(x + 1\right)}^{2} + 2 \, g_{n} {\left(x + 1\right)} + f_{n} + g_{n} =\\
 {\left(4 \, f_{n} + 4 \, g_{n}\right)} {\left(x + 2\right)}^{4} + {\left(4 \, f_{n} + 3 \, g_{n}\right)} {\left(x + 2\right)}^{3} + 2 \, g_{n} {\left(x + 2\right)}^{2} + {\left(2 \, f_{n} + 3 \, g_{n}\right)} {\left(x + 2\right)} + f_{n} + 4 \, g_{n} =\\
 {\left(4 \, f_{n} + 4 \, g_{n}\right)} {\left(x + 3\right)}^{4} + {\left(3 \, f_{n} + 2 \, g_{n}\right)} {\left(x + 3\right)}^{3} + 2 \, {\left(f_{n} + g_{n}\right)} {\left(x + 3\right)}^{2} + {\left(3 \, f_{n} + 2 \, g_{n}\right)} {\left(x + 3\right)} + 4 \, f_{n} + 4 \, g_{n} =\\
 {\left(4 \, f_{n} + 4 \, g_{n}\right)} {\left(x + 4\right)}^{4} + {\left(2 \, f_{n} + g_{n}\right)} {\left(x + 4\right)}^{3} + 2 \, f_{n} {\left(x + 4\right)}^{2} + {\left(2 \, f_{n} + 3 \, g_{n}\right)} {\left(x + 4\right)} + 4 \, f_{n} + g_{n} =\\
 \end{array}$$
\end{myth}

 \begin{proof}

Первое равенство следует из теоремы 1.

При поляризации $x_{n+1}$, когда $d_{n+1} = 0$
$$\begin{array}{l}
s_{n+1}^1(\bar{x}_n, 0) = 0 + 0 + 0 + 0 + f_n + g_n = f_n + g_n \\
s_{n+1}^1(\bar{x}_n, 1) = 4\,f_n + 4\,g_n + f_n + g_n + 3\,f_n + f_n + g_n = 4\,f_n + g_n \\
s_{n+1}^1(\bar{x}_n, 2) = 4\,f_n + 4\,g_n + 3\,f_n + 4\,g_n + f_n + f_n + g_n = 4\,f_n + 4\,g_n \\
s_{n+1}^1(\bar{x}_n, 3) = 4\,f_n + 4\,g_n + 2\,f_n + 4\,g_n + 4\,f_n + f_n + g_n = f_n + 4\,g_n \\
s_{n+1}^1(\bar{x}_n, 4) = 4\,f_n + 4\,g_n + 4\,f_n + g_n + 2\,f_n + f_n + g_n = f_n + g_n \\
\end{array}$$
При поляризации $x_{n+1}$, когда $d_{n+1} = 1$
$$\begin{array}{l}
s_{n+1}^1(\bar{x}_n, 0) = 4\,f_n + 4\,g_n + 4\,g_n + f_n + 2\,g_n + f_n + g_n = f_n + g_n \\
s_{n+1}^1(\bar{x}_n, 1) = 4\,f_n + 4\,g_n + 2\,g_n + 4\,f_n + 4\,g_n + f_n + g_n = 4\,f_n + g_n \\
s_{n+1}^1(\bar{x}_n, 2) = 4\,f_n + 4\,g_n + 3\,g_n + 4\,f_n + g_n + f_n + g_n = 4\,f_n + 4\,g_n \\
s_{n+1}^1(\bar{x}_n, 3) = 4\,f_n + 4\,g_n + g_n + f_n + 3\,g_n + f_n + g_n = f_n + 4\,g_n \\
s_{n+1}^1(\bar{x}_n, 4) = 0 + 0 + 0 + 0 + f_n + g_n = f_n + g_n \\
\end{array}$$
При поляризации $x_{n+1}$, когда $d_{n+1} = 2$
$$\begin{array}{l}
s_{n+1}^1(\bar{x}_n, 0) = 4\,f_n + 4\,g_n + 2\,f_n + 4\,g_n + 3\,g_n + 4\,f_n + g_n + f_n + 4\,g_n = f_n + g_n \\
s_{n+1}^1(\bar{x}_n, 1) = 4\,f_n + 4\,g_n + 3\,f_n + g_n + 3\,g_n + f_n + 4\,g_n + f_n + 4\,g_n = 4\,f_n + g_n \\
s_{n+1}^1(\bar{x}_n, 2) = 4\,f_n + 4\,g_n + f_n + 2\,g_n + 2\,g_n + 3\,f_n + 2\,g_n + f_n + 4\,g_n = 4\,f_n + 4\,g_n \\
s_{n+1}^1(\bar{x}_n, 3) = 0 + 0 + 0 + 0 + f_n + 4\,g_n = f_n + 4\,g_n \\
s_{n+1}^1(\bar{x}_n, 4) = 4\,f_n + 4\,g_n + 4\,f_n + 3\,g_n + 2\,g_n + 2\,f_n + 3\,g_n + f_n + 4\,g_n = f_n + g_n \\
\end{array}$$
При поляризации $x_{n+1}$, когда $d_{n+1} = 3$
$$\begin{array}{l}
s_{n+1}^1(\bar{x}_n, 0) = 4\,f_n + 4\,g_n + f_n + 4\,g_n + 3\,f_n + 3\,g_n + 4\,f_n + g_n + 4\,f_n + 4\,g_n = f_n + g_n \\
s_{n+1}^1(\bar{x}_n, 1) = 4\,f_n + 4\,g_n + 2\,f_n + 3\,g_n + 2\,f_n + 2\,g_n + 2\,f_n + 3\,g_n + 4\,f_n + 4\,g_n = 4\,f_n + g_n \\
s_{n+1}^1(\bar{x}_n, 2) = 0 + 0 + 0 + 0 + 4\,f_n + 4\,g_n = 4\,f_n + 4\,g_n \\
s_{n+1}^1(\bar{x}_n, 3) = 4\,f_n + 4\,g_n + 3\,f_n + 2\,g_n + 2\,f_n + 2\,g_n + 3\,f_n + 2\,g_n + 4\,f_n + 4\,g_n = f_n + 4\,g_n \\
s_{n+1}^1(\bar{x}_n, 4) = 4\,f_n + 4\,g_n + 4\,f_n + g_n + 3\,f_n + 3\,g_n + f_n + 4\,g_n + 4\,f_n + 4\,g_n = f_n + g_n \\
\end{array}$$
При поляризации $x_{n+1}$, когда $d_{n+1} = 4$
$$\begin{array}{l}
s_{n+1}^1(\bar{x}_n, 0) = 4\,f_n + 4\,g_n + 3\,f_n + 4\,g_n + 2\,f_n + 3\,f_n + 2\,g_n + 4\,f_n + g_n = f_n + g_n \\
s_{n+1}^1(\bar{x}_n, 1) = 0 + 0 + 0 + 0 + 4\,f_n + g_n = 4\,f_n + g_n \\
s_{n+1}^1(\bar{x}_n, 2) = 4\,f_n + 4\,g_n + 2\,f_n + g_n + 2\,f_n + 2\,f_n + 3\,g_n + 4\,f_n + g_n = 4\,f_n + 4\,g_n \\
s_{n+1}^1(\bar{x}_n, 3) = 4\,f_n + 4\,g_n + f_n + 3\,g_n + 3\,f_n + 4\,f_n + g_n + 4\,f_n + g_n = f_n + 4\,g_n \\
s_{n+1}^1(\bar{x}_n, 4) = 4\,f_n + 4\,g_n + 4\,f_n + 2\,g_n + 3\,f_n + f_n + 4\,g_n + 4\,f_n + g_n = f_n + g_n \\
\end{array}$$

\end{proof}

\begin{myth} При $n \geqslant 1 $ для периодических функций пятизначной логики $f_n = f^{\left(n\right)}_{\left(1144\right)}$,
$g_n = f^{\left(n\right)}_{\left(1441\right)}$ верны следующие равенства:
$$\begin{array}{l}
s_{n+1}^2 = f_{n+1} + 2\,g_{n+1}=\\
 {\left(4 \, f_{n} + 3 \, g_{n}\right)} x^{4} + {\left(4 \, f_{n} + 3 \, g_{n}\right)} x^{3} + {\left(2 \, f_{n} + 4 \, g_{n}\right)} x^{2} + {\left(2 \, f_{n} + 4 \, g_{n}\right)} x + f_{n} + 2 \, g_{n} =\\
 {\left(4 \, f_{n} + 3 \, g_{n}\right)} {\left(x + 1\right)}^{4} + {\left(3 \, f_{n} + g_{n}\right)} {\left(x + 1\right)}^{3} + {\left(4 \, f_{n} + 3 \, g_{n}\right)} {\left(x + 1\right)}^{2} + {\left(4 \, f_{n} + 3 \, g_{n}\right)} {\left(x + 1\right)} + f_{n} + 2 \, g_{n} =\\
 {\left(4 \, f_{n} + 3 \, g_{n}\right)} {\left(x + 2\right)}^{4} + {\left(2 \, f_{n} + 4 \, g_{n}\right)} {\left(x + 2\right)}^{3} + {\left(4 \, f_{n} + 3 \, g_{n}\right)} {\left(x + 2\right)}^{2} + {\left(4 \, f_{n} + 3 \, g_{n}\right)} {\left(x + 2\right)} + 2 \, f_{n} + 4 \, g_{n} =\\
 {\left(4 \, f_{n} + 3 \, g_{n}\right)} {\left(x + 3\right)}^{4} + {\left(f_{n} + 2 \, g_{n}\right)} {\left(x + 3\right)}^{3} + 2 \, {\left(f_{n} + 2 \, g_{n}\right)} {\left(x + 3\right)}^{2} + {\left(f_{n} + 2 \, g_{n}\right)} {\left(x + 3\right)} + 4 \, f_{n} + 3 \, g_{n} =\\
 {\left(4 \, f_{n} + 3 \, g_{n}\right)} {\left(x + 4\right)}^{4} + {\left(3 \, f_{n} + g_{n}\right)} {\left(x + 4\right)}^{2} + {\left(4 \, f_{n} + 3 \, g_{n}\right)} {\left(x + 4\right)} + 3 \, f_{n} + g_{n} =\\
 \end{array}$$
\end{myth}

 \begin{proof}

Первое равенство следует из теоремы 1.

При поляризации $x_{n+1}$, когда $d_{n+1} = 0$
$$\begin{array}{l}
s_{n+1}^2(\bar{x}_n, 0) = 0 + 0 + 0 + 0 + f_n + 2\,g_n = f_n + 2\,g_n \\
s_{n+1}^2(\bar{x}_n, 1) = 4\,f_n + 3\,g_n + 4\,f_n + 3\,g_n + 2\,f_n + 4\,g_n + 2\,f_n + 4\,g_n + f_n + 2\,g_n = 3\,f_n + g_n \\
s_{n+1}^2(\bar{x}_n, 2) = 4\,f_n + 3\,g_n + 2\,f_n + 4\,g_n + 3\,f_n + g_n + 4\,f_n + 3\,g_n + f_n + 2\,g_n = 4\,f_n + 3\,g_n \\
s_{n+1}^2(\bar{x}_n, 3) = 4\,f_n + 3\,g_n + 3\,f_n + g_n + 3\,f_n + g_n + f_n + 2\,g_n + f_n + 2\,g_n = 2\,f_n + 4\,g_n \\
s_{n+1}^2(\bar{x}_n, 4) = 4\,f_n + 3\,g_n + f_n + 2\,g_n + 2\,f_n + 4\,g_n + 3\,f_n + g_n + f_n + 2\,g_n = f_n + 2\,g_n \\
\end{array}$$
При поляризации $x_{n+1}$, когда $d_{n+1} = 1$
$$\begin{array}{l}
s_{n+1}^2(\bar{x}_n, 0) = 4\,f_n + 3\,g_n + 3\,f_n + g_n + 4\,f_n + 3\,g_n + 4\,f_n + 3\,g_n + f_n + 2\,g_n = f_n + 2\,g_n \\
s_{n+1}^2(\bar{x}_n, 1) = 4\,f_n + 3\,g_n + 4\,f_n + 3\,g_n + f_n + 2\,g_n + 3\,f_n + g_n + f_n + 2\,g_n = 3\,f_n + g_n \\
s_{n+1}^2(\bar{x}_n, 2) = 4\,f_n + 3\,g_n + f_n + 2\,g_n + f_n + 2\,g_n + 2\,f_n + 4\,g_n + f_n + 2\,g_n = 4\,f_n + 3\,g_n \\
s_{n+1}^2(\bar{x}_n, 3) = 4\,f_n + 3\,g_n + 2\,f_n + 4\,g_n + 4\,f_n + 3\,g_n + f_n + 2\,g_n + f_n + 2\,g_n = 2\,f_n + 4\,g_n \\
s_{n+1}^2(\bar{x}_n, 4) = 0 + 0 + 0 + 0 + f_n + 2\,g_n = f_n + 2\,g_n \\
\end{array}$$
При поляризации $x_{n+1}$, когда $d_{n+1} = 2$
$$\begin{array}{l}
s_{n+1}^2(\bar{x}_n, 0) = 4\,f_n + 3\,g_n + f_n + 2\,g_n + f_n + 2\,g_n + 3\,f_n + g_n + 2\,f_n + 4\,g_n = f_n + 2\,g_n \\
s_{n+1}^2(\bar{x}_n, 1) = 4\,f_n + 3\,g_n + 4\,f_n + 3\,g_n + f_n + 2\,g_n + 2\,f_n + 4\,g_n + 2\,f_n + 4\,g_n = 3\,f_n + g_n \\
s_{n+1}^2(\bar{x}_n, 2) = 4\,f_n + 3\,g_n + 3\,f_n + g_n + 4\,f_n + 3\,g_n + f_n + 2\,g_n + 2\,f_n + 4\,g_n = 4\,f_n + 3\,g_n \\
s_{n+1}^2(\bar{x}_n, 3) = 0 + 0 + 0 + 0 + 2\,f_n + 4\,g_n = 2\,f_n + 4\,g_n \\
s_{n+1}^2(\bar{x}_n, 4) = 4\,f_n + 3\,g_n + 2\,f_n + 4\,g_n + 4\,f_n + 3\,g_n + 4\,f_n + 3\,g_n + 2\,f_n + 4\,g_n = f_n + 2\,g_n \\
\end{array}$$
При поляризации $x_{n+1}$, когда $d_{n+1} = 3$
$$\begin{array}{l}
s_{n+1}^2(\bar{x}_n, 0) = 4\,f_n + 3\,g_n + 2\,f_n + 4\,g_n + 3\,f_n + g_n + 3\,f_n + g_n + 4\,f_n + 3\,g_n = f_n + 2\,g_n \\
s_{n+1}^2(\bar{x}_n, 1) = 4\,f_n + 3\,g_n + 4\,f_n + 3\,g_n + 2\,f_n + 4\,g_n + 4\,f_n + 3\,g_n + 4\,f_n + 3\,g_n = 3\,f_n + g_n \\
s_{n+1}^2(\bar{x}_n, 2) = 0 + 0 + 0 + 0 + 4\,f_n + 3\,g_n = 4\,f_n + 3\,g_n \\
s_{n+1}^2(\bar{x}_n, 3) = 4\,f_n + 3\,g_n + f_n + 2\,g_n + 2\,f_n + 4\,g_n + f_n + 2\,g_n + 4\,f_n + 3\,g_n = 2\,f_n + 4\,g_n \\
s_{n+1}^2(\bar{x}_n, 4) = 4\,f_n + 3\,g_n + 3\,f_n + g_n + 3\,f_n + g_n + 2\,f_n + 4\,g_n + 4\,f_n + 3\,g_n = f_n + 2\,g_n \\
\end{array}$$
При поляризации $x_{n+1}$, когда $d_{n+1} = 4$
$$\begin{array}{l}
s_{n+1}^2(\bar{x}_n, 0) = 4\,f_n + 3\,g_n + 0 + 3\,f_n + g_n + f_n + 2\,g_n + 3\,f_n + g_n = f_n + 2\,g_n \\
s_{n+1}^2(\bar{x}_n, 1) = 0 + 0 + 0 + 0 + 3\,f_n + g_n = 3\,f_n + g_n \\
s_{n+1}^2(\bar{x}_n, 2) = 4\,f_n + 3\,g_n + 0 + 3\,f_n + g_n + 4\,f_n + 3\,g_n + 3\,f_n + g_n = 4\,f_n + 3\,g_n \\
s_{n+1}^2(\bar{x}_n, 3) = 4\,f_n + 3\,g_n + 0 + 2\,f_n + 4\,g_n + 3\,f_n + g_n + 3\,f_n + g_n = 2\,f_n + 4\,g_n \\
s_{n+1}^2(\bar{x}_n, 4) = 4\,f_n + 3\,g_n + 0 + 2\,f_n + 4\,g_n + 2\,f_n + 4\,g_n + 3\,f_n + g_n = f_n + 2\,g_n \\
\end{array}$$

\end{proof}

\begin{myth} При $n \geqslant 1 $ для периодических функций пятизначной логики $f_n = f^{\left(n\right)}_{\left(1144\right)}$,
$g_n = f^{\left(n\right)}_{\left(1441\right)}$ верны следующие равенства:
$$\begin{array}{l}
s_{n+1}^3 = f_{n+1} + 3\,g_{n+1}=\\
 {\left(4 \, f_{n} + 2 \, g_{n}\right)} x^{4} + {\left(2 \, f_{n} + g_{n}\right)} x^{3} + {\left(4 \, f_{n} + 2 \, g_{n}\right)} x^{2} + {\left(f_{n} + 3 \, g_{n}\right)} x + f_{n} + 3 \, g_{n} =\\
 {\left(4 \, f_{n} + 2 \, g_{n}\right)} {\left(x + 1\right)}^{4} + {\left(f_{n} + 3 \, g_{n}\right)} {\left(x + 1\right)}^{3} + {\left(2 \, f_{n} + g_{n}\right)} {\left(x + 1\right)}^{2} + {\left(3 \, f_{n} + 4 \, g_{n}\right)} {\left(x + 1\right)} + f_{n} + 3 \, g_{n} =\\
 {\left(4 \, f_{n} + 2 \, g_{n}\right)} {\left(x + 2\right)}^{4} + {\left(3 \, f_{n} + 4 \, g_{n}\right)} {\left(x + 2\right)}^{2} + {\left(f_{n} + 3 \, g_{n}\right)} {\left(x + 2\right)} + 3 \, f_{n} + 4 \, g_{n} =\\
 {\left(4 \, f_{n} + 2 \, g_{n}\right)} {\left(x + 3\right)}^{4} + {\left(4 \, f_{n} + 2 \, g_{n}\right)} {\left(x + 3\right)}^{3} + 2 \, {\left(f_{n} + 3 \, g_{n}\right)} {\left(x + 3\right)}^{2} + {\left(4 \, f_{n} + 2 \, g_{n}\right)} {\left(x + 3\right)} + 4 \, f_{n} + 2 \, g_{n} =\\
 {\left(4 \, f_{n} + 2 \, g_{n}\right)} {\left(x + 4\right)}^{4} + {\left(3 \, f_{n} + 4 \, g_{n}\right)} {\left(x + 4\right)}^{3} + {\left(4 \, f_{n} + 2 \, g_{n}\right)} {\left(x + 4\right)}^{2} + {\left(f_{n} + 3 \, g_{n}\right)} {\left(x + 4\right)} + 2 \, f_{n} + g_{n} =\\
 \end{array}$$
\end{myth}

 \begin{proof}

Первое равенство следует из теоремы 1.

При поляризации $x_{n+1}$, когда $d_{n+1} = 0$
$$\begin{array}{l}
s_{n+1}^3(\bar{x}_n, 0) = 0 + 0 + 0 + 0 + f_n + 3\,g_n = f_n + 3\,g_n \\
s_{n+1}^3(\bar{x}_n, 1) = 4\,f_n + 2\,g_n + 2\,f_n + g_n + 4\,f_n + 2\,g_n + f_n + 3\,g_n + f_n + 3\,g_n = 2\,f_n + g_n \\
s_{n+1}^3(\bar{x}_n, 2) = 4\,f_n + 2\,g_n + f_n + 3\,g_n + f_n + 3\,g_n + 2\,f_n + g_n + f_n + 3\,g_n = 4\,f_n + 2\,g_n \\
s_{n+1}^3(\bar{x}_n, 3) = 4\,f_n + 2\,g_n + 4\,f_n + 2\,g_n + f_n + 3\,g_n + 3\,f_n + 4\,g_n + f_n + 3\,g_n = 3\,f_n + 4\,g_n \\
s_{n+1}^3(\bar{x}_n, 4) = 4\,f_n + 2\,g_n + 3\,f_n + 4\,g_n + 4\,f_n + 2\,g_n + 4\,f_n + 2\,g_n + f_n + 3\,g_n = f_n + 3\,g_n \\
\end{array}$$
При поляризации $x_{n+1}$, когда $d_{n+1} = 1$
$$\begin{array}{l}
s_{n+1}^3(\bar{x}_n, 0) = 4\,f_n + 2\,g_n + f_n + 3\,g_n + 2\,f_n + g_n + 3\,f_n + 4\,g_n + f_n + 3\,g_n = f_n + 3\,g_n \\
s_{n+1}^3(\bar{x}_n, 1) = 4\,f_n + 2\,g_n + 3\,f_n + 4\,g_n + 3\,f_n + 4\,g_n + f_n + 3\,g_n + f_n + 3\,g_n = 2\,f_n + g_n \\
s_{n+1}^3(\bar{x}_n, 2) = 4\,f_n + 2\,g_n + 2\,f_n + g_n + 3\,f_n + 4\,g_n + 4\,f_n + 2\,g_n + f_n + 3\,g_n = 4\,f_n + 2\,g_n \\
s_{n+1}^3(\bar{x}_n, 3) = 4\,f_n + 2\,g_n + 4\,f_n + 2\,g_n + 2\,f_n + g_n + 2\,f_n + g_n + f_n + 3\,g_n = 3\,f_n + 4\,g_n \\
s_{n+1}^3(\bar{x}_n, 4) = 0 + 0 + 0 + 0 + f_n + 3\,g_n = f_n + 3\,g_n \\
\end{array}$$
При поляризации $x_{n+1}$, когда $d_{n+1} = 2$
$$\begin{array}{l}
s_{n+1}^3(\bar{x}_n, 0) = 4\,f_n + 2\,g_n + 0 + 2\,f_n + g_n + 2\,f_n + g_n + 3\,f_n + 4\,g_n = f_n + 3\,g_n \\
s_{n+1}^3(\bar{x}_n, 1) = 4\,f_n + 2\,g_n + 0 + 2\,f_n + g_n + 3\,f_n + 4\,g_n + 3\,f_n + 4\,g_n = 2\,f_n + g_n \\
s_{n+1}^3(\bar{x}_n, 2) = 4\,f_n + 2\,g_n + 0 + 3\,f_n + 4\,g_n + 4\,f_n + 2\,g_n + 3\,f_n + 4\,g_n = 4\,f_n + 2\,g_n \\
s_{n+1}^3(\bar{x}_n, 3) = 0 + 0 + 0 + 0 + 3\,f_n + 4\,g_n = 3\,f_n + 4\,g_n \\
s_{n+1}^3(\bar{x}_n, 4) = 4\,f_n + 2\,g_n + 0 + 3\,f_n + 4\,g_n + f_n + 3\,g_n + 3\,f_n + 4\,g_n = f_n + 3\,g_n \\
\end{array}$$
При поляризации $x_{n+1}$, когда $d_{n+1} = 3$
$$\begin{array}{l}
s_{n+1}^3(\bar{x}_n, 0) = 4\,f_n + 2\,g_n + 3\,f_n + 4\,g_n + 3\,f_n + 4\,g_n + 2\,f_n + g_n + 4\,f_n + 2\,g_n = f_n + 3\,g_n \\
s_{n+1}^3(\bar{x}_n, 1) = 4\,f_n + 2\,g_n + f_n + 3\,g_n + 2\,f_n + g_n + f_n + 3\,g_n + 4\,f_n + 2\,g_n = 2\,f_n + g_n \\
s_{n+1}^3(\bar{x}_n, 2) = 0 + 0 + 0 + 0 + 4\,f_n + 2\,g_n = 4\,f_n + 2\,g_n \\
s_{n+1}^3(\bar{x}_n, 3) = 4\,f_n + 2\,g_n + 4\,f_n + 2\,g_n + 2\,f_n + g_n + 4\,f_n + 2\,g_n + 4\,f_n + 2\,g_n = 3\,f_n + 4\,g_n \\
s_{n+1}^3(\bar{x}_n, 4) = 4\,f_n + 2\,g_n + 2\,f_n + g_n + 3\,f_n + 4\,g_n + 3\,f_n + 4\,g_n + 4\,f_n + 2\,g_n = f_n + 3\,g_n \\
\end{array}$$
При поляризации $x_{n+1}$, когда $d_{n+1} = 4$
$$\begin{array}{l}
s_{n+1}^3(\bar{x}_n, 0) = 4\,f_n + 2\,g_n + 2\,f_n + g_n + 4\,f_n + 2\,g_n + 4\,f_n + 2\,g_n + 2\,f_n + g_n = f_n + 3\,g_n \\
s_{n+1}^3(\bar{x}_n, 1) = 0 + 0 + 0 + 0 + 2\,f_n + g_n = 2\,f_n + g_n \\
s_{n+1}^3(\bar{x}_n, 2) = 4\,f_n + 2\,g_n + 3\,f_n + 4\,g_n + 4\,f_n + 2\,g_n + f_n + 3\,g_n + 2\,f_n + g_n = 4\,f_n + 2\,g_n \\
s_{n+1}^3(\bar{x}_n, 3) = 4\,f_n + 2\,g_n + 4\,f_n + 2\,g_n + f_n + 3\,g_n + 2\,f_n + g_n + 2\,f_n + g_n = 3\,f_n + 4\,g_n \\
s_{n+1}^3(\bar{x}_n, 4) = 4\,f_n + 2\,g_n + f_n + 3\,g_n + f_n + 3\,g_n + 3\,f_n + 4\,g_n + 2\,f_n + g_n = f_n + 3\,g_n \\
\end{array}$$

\end{proof}

\begin{myth} При $n \geqslant 1 $ для периодических функций пятизначной логики $f_n = f^{\left(n\right)}_{\left(1144\right)}$,
$g_n = f^{\left(n\right)}_{\left(1441\right)}$ верны следующие равенства:
$$\begin{array}{l}
s_{n+1}^4 = f_{n+1} + 4\,g_{n+1}=\\
 {\left(4 \, f_{n} + g_{n}\right)} x^{4} + 4 \, g_{n} x^{3} + f_{n} x^{2} + 2 \, g_{n} x + f_{n} + 4 \, g_{n} =\\
 {\left(4 \, f_{n} + g_{n}\right)} {\left(x + 1\right)}^{4} + 4 \, f_{n} {\left(x + 1\right)}^{3} + 4 \, g_{n} {\left(x + 1\right)}^{2} + 2 \, f_{n} {\left(x + 1\right)} + 5 \, g_{n} {\left(x + 1\right)} + f_{n} + 4 \, g_{n} =\\
 {\left(4 \, f_{n} + g_{n}\right)} {\left(x + 2\right)}^{4} + {\left(3 \, f_{n} + g_{n}\right)} {\left(x + 2\right)}^{3} + 2 \, f_{n} {\left(x + 2\right)}^{2} + {\left(3 \, f_{n} + 3 \, g_{n}\right)} {\left(x + 2\right)} + 4 \, f_{n} + 4 \, g_{n} =\\
 {\left(4 \, f_{n} + g_{n}\right)} {\left(x + 3\right)}^{4} + {\left(2 \, f_{n} + 2 \, g_{n}\right)} {\left(x + 3\right)}^{3} + 2 \, {\left(f_{n} + 4 \, g_{n}\right)} {\left(x + 3\right)}^{2} + {\left(2 \, f_{n} + 2 \, g_{n}\right)} {\left(x + 3\right)} + 4 \, f_{n} + g_{n} =\\
 {\left(4 \, f_{n} + g_{n}\right)} {\left(x + 4\right)}^{4} + {\left(f_{n} + 3 \, g_{n}\right)} {\left(x + 4\right)}^{3} + 5 \, f_{n} {\left(x + 4\right)}^{2} + 3 \, g_{n} {\left(x + 4\right)}^{2} + {\left(3 \, f_{n} + 3 \, g_{n}\right)} {\left(x + 4\right)} + f_{n} + g_{n} =\\
 \end{array}$$
\end{myth}

 \begin{proof}

Первое равенство следует из теоремы 1.

При поляризации $x_{n+1}$, когда $d_{n+1} = 0$
$$\begin{array}{l}
s_{n+1}^4(\bar{x}_n, 0) = 0 + 0 + 0 + 0 + f_n + 4\,g_n = f_n + 4\,g_n \\
s_{n+1}^4(\bar{x}_n, 1) = 4\,f_n + g_n + 4\,g_n + f_n + 2\,g_n + f_n + 4\,g_n = f_n + g_n \\
s_{n+1}^4(\bar{x}_n, 2) = 4\,f_n + g_n + 2\,g_n + 4\,f_n + 4\,g_n + f_n + 4\,g_n = 4\,f_n + g_n \\
s_{n+1}^4(\bar{x}_n, 3) = 4\,f_n + g_n + 3\,g_n + 4\,f_n + g_n + f_n + 4\,g_n = 4\,f_n + 4\,g_n \\
s_{n+1}^4(\bar{x}_n, 4) = 4\,f_n + g_n + g_n + f_n + 3\,g_n + f_n + 4\,g_n = f_n + 4\,g_n \\
\end{array}$$
При поляризации $x_{n+1}$, когда $d_{n+1} = 1$
$$\begin{array}{l}
s_{n+1}^4(\bar{x}_n, 0) = 4\,f_n + g_n + 4\,f_n + 4\,g_n + 2\,f_n + 0\,g_n + f_n + 4\,g_n = f_n + 4\,g_n \\
s_{n+1}^4(\bar{x}_n, 1) = 4\,f_n + g_n + 2\,f_n + g_n + 4\,f_n + 0\,g_n + f_n + 4\,g_n = f_n + g_n \\
s_{n+1}^4(\bar{x}_n, 2) = 4\,f_n + g_n + 3\,f_n + g_n + f_n + 0\,g_n + f_n + 4\,g_n = 4\,f_n + g_n \\
s_{n+1}^4(\bar{x}_n, 3) = 4\,f_n + g_n + f_n + 4\,g_n + 3\,f_n + 0\,g_n + f_n + 4\,g_n = 4\,f_n + 4\,g_n \\
s_{n+1}^4(\bar{x}_n, 4) = 0 + 0 + 0 + 0 + f_n + 4\,g_n = f_n + 4\,g_n \\
\end{array}$$
При поляризации $x_{n+1}$, когда $d_{n+1} = 2$
$$\begin{array}{l}
s_{n+1}^4(\bar{x}_n, 0) = 4\,f_n + g_n + 4\,f_n + 3\,g_n + 3\,f_n + f_n + g_n + 4\,f_n + 4\,g_n = f_n + 4\,g_n \\
s_{n+1}^4(\bar{x}_n, 1) = 4\,f_n + g_n + f_n + 2\,g_n + 3\,f_n + 4\,f_n + 4\,g_n + 4\,f_n + 4\,g_n = f_n + g_n \\
s_{n+1}^4(\bar{x}_n, 2) = 4\,f_n + g_n + 2\,f_n + 4\,g_n + 2\,f_n + 2\,f_n + 2\,g_n + 4\,f_n + 4\,g_n = 4\,f_n + g_n \\
s_{n+1}^4(\bar{x}_n, 3) = 0 + 0 + 0 + 0 + 4\,f_n + 4\,g_n = 4\,f_n + 4\,g_n \\
s_{n+1}^4(\bar{x}_n, 4) = 4\,f_n + g_n + 3\,f_n + g_n + 2\,f_n + 3\,f_n + 3\,g_n + 4\,f_n + 4\,g_n = f_n + 4\,g_n \\
\end{array}$$
При поляризации $x_{n+1}$, когда $d_{n+1} = 3$
$$\begin{array}{l}
s_{n+1}^4(\bar{x}_n, 0) = 4\,f_n + g_n + 4\,f_n + 4\,g_n + 3\,f_n + 2\,g_n + f_n + g_n + 4\,f_n + g_n = f_n + 4\,g_n \\
s_{n+1}^4(\bar{x}_n, 1) = 4\,f_n + g_n + 3\,f_n + 3\,g_n + 2\,f_n + 3\,g_n + 3\,f_n + 3\,g_n + 4\,f_n + g_n = f_n + g_n \\
s_{n+1}^4(\bar{x}_n, 2) = 0 + 0 + 0 + 0 + 4\,f_n + g_n = 4\,f_n + g_n \\
s_{n+1}^4(\bar{x}_n, 3) = 4\,f_n + g_n + 2\,f_n + 2\,g_n + 2\,f_n + 3\,g_n + 2\,f_n + 2\,g_n + 4\,f_n + g_n = 4\,f_n + 4\,g_n \\
s_{n+1}^4(\bar{x}_n, 4) = 4\,f_n + g_n + f_n + g_n + 3\,f_n + 2\,g_n + 4\,f_n + 4\,g_n + 4\,f_n + g_n = f_n + 4\,g_n \\
\end{array}$$
При поляризации $x_{n+1}$, когда $d_{n+1} = 4$
$$\begin{array}{l}
s_{n+1}^4(\bar{x}_n, 0) = 4\,f_n + g_n + 4\,f_n + 2\,g_n + 0\,f_n + 3\,g_n + 2\,f_n + 2\,g_n + f_n + g_n = f_n + 4\,g_n \\
s_{n+1}^4(\bar{x}_n, 1) = 0 + 0 + 0 + 0 + f_n + g_n = f_n + g_n \\
s_{n+1}^4(\bar{x}_n, 2) = 4\,f_n + g_n + f_n + 3\,g_n + 0\,f_n + 3\,g_n + 3\,f_n + 3\,g_n + f_n + g_n = 4\,f_n + g_n \\
s_{n+1}^4(\bar{x}_n, 3) = 4\,f_n + g_n + 3\,f_n + 4\,g_n + 0\,f_n + 2\,g_n + f_n + g_n + f_n + g_n = 4\,f_n + 4\,g_n \\
s_{n+1}^4(\bar{x}_n, 4) = 4\,f_n + g_n + 2\,f_n + g_n + 0\,f_n + 2\,g_n + 4\,f_n + 4\,g_n + f_n + g_n = f_n + 4\,g_n \\
\end{array}$$

\end{proof}



\makeatletter
\renewcommand*{\@biblabel}[1]{\hfill#1.}
\makeatother

\newpage

\begin{thebibliography}{0}
\bibitem{ue04} Угрюмов~Е.\,П. Цифровая схемотехника. СПб.: БХВ-Петербург, 2004.
\bibitem{sb90} Sasao T., Besslich P. On the complexity of mod-2 sum PLA’s  // IEEE Trans.on Comput. 39. N 2. 1990. P.~262--266. 
\bibitem{sv93} Супрун~В.\,П. Сложность булевых функций в классе канонических поляризованных полиномов // Дискретная математика. 5.
    \textnumero 2. 1993. С. 111--115. 
\bibitem{pn95} Перязев Н.\,А. Сложность булевых функций в классе полиномиальных поляризованных~форм // Алгебра и логика. 34.
    \textnumero 3. 1995. С. 323--326. 
\bibitem{ss02} Селезнева С.\,H. О сложности представления функций многозначных логик поляризованными полиномами. Дискретная
    математика. 14. \textnumero 2. 2002. С.~48--53.
\bibitem{kk05} Кириченко~К.\,Д. Верхняя оценка сложности полиномиальных нормальных форм булевых функций 
    // Дискретная математика. 17. \textnumero 3. 2005. С. 80--88.
\bibitem{sd08} Селезнева С.\,Н. Дайняк А.\,Б. О сложности обобщенных полиномов k\nobreakdash-значных функций // Вестник Московского
    университета. Серия 15. Вычислительная математика и кибернетика. \textnumero 3. 2008. С. 34--39.
\bibitem{mn12} Маркелов Н.\,К. Нижняя оценка сложности функций трехзначной логики в классе поляризованных полиномов // Вестник
    Московского университета. Серия 15. Вычислительная математика и кибернетика. \textnumero 3. 2012. С. 40--45.
\bibitem{sm09} Селезнева С.\,H. Маркелов Н.\,К. Быстрый алгоритм построения векторов коэффициэнтов поляризованных полиномов
    k-значных функций // Ученые записки Казанского университета. Серия Физико-математические науки. 2009. 151.
    \textnumero 2 С.~147-151.
 
\end{thebibliography}

\end{document}
