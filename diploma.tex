\documentclass[a4paper, 14pt]{extarticle}
\usepackage[utf8]{inputenc}
\usepackage[russian]{babel}
\usepackage{longtable, moreverb}
\usepackage{ amssymb, latexsym, amsmath}

\sloppy

\begin{document}

%\tableofcontents

%\newpage

\setcounter{secnumdepth}{-1}

\section{Введение}
Одним из стандартных способов задания функций k\nobreakdash-значной логики являются поляризованные полиномиальные формы(ППФ),
которые также называются обобщенными формами Рида-Мюллера, или каноническими поляризованными полиномами.
Длиной полиномиальной формы называется число слагаемых в ней. Для функции k\nobreakdash-значной логики $F$ сложностью в классе
поляризованных полиномиальных форм называется длина кратчайшей ПНФ, реализующей~$F$.
Функция Шеннона длины $L_k(n)$ определяется как наибольшая длина среди всех функций k\nobreakdash-значной логики от~$n$~переменных.

Практическое применение ППФ нашли при построении программируемых логических матриц~\cite{ue04}, сложность которых
напрямую зависит от длины ППФ.

В 1993  В. П. Супрун \cite{sv93} получил следующие оценки функции Шеннона для булевых функций :
$$
L_2(n) \geqslant C_n^{[\frac{n}{2}]},
$$
$$
L_2(n) < 3 \cdot 2^{n-1}.
$$
где [$\cdot$] обозначает целую часть.

Точное значение функции Шеннона для булевых функций в 1995 году было
найдено Н. А. Перязевым~\cite{pn95} :
$$
L_2(n) = \left[\frac{2^{n+1}}{3}\right],
$$

Для функций k\nobreakdash-значной логики верхняя оценка функции Шеннона была получена в 2002 году С. Н. Селезневой~\cite{ss02} :
$$
L_k(n) = \frac{k(k-1)}{k(k-1)+1}k^n.
$$

Рассматриваются и другие полиномиальные формы. Например класс обобщенных полиномиальных форм.
В классе обобщенных полиномиальных форм, в отличие от класса поляризованных полиномиальных форм, переменные могут иметь
различную поляризацию в разных слагаемых. В статье К. Д. Кириченко \cite{kk05}, опубликованной в 2005 году, получена верхняя оценка
функции Шеннона в классе обобщенных полиномиальных форм булевых функций:
$$
L^{\text{О.П.}}_2(n) < \frac{2 ^ {n + 1}(\log_2n+1)}{n}.
$$

Верхняя оценка функции Шеннона в классе обобщенных полиномиальных форм функций k\nobreakdash-значной логики была получена
Селезневой С.\,Н. Дайняком А.\,Б в 2008 году \cite{sd08}:
$$
L^{\text{О.П.}}_k(n) \lesssim 2\cdot\frac{k ^ n}{n}\cdot \ln n \text{ при } n \rightarrow \infty.
$$

С. Н. Селезневой и Н. К. Маркеловым в 2009 году \cite{sm09} был получен алгоритм быстрого нахождения коэффициэнтов
ППФ в k\nobreakdash-значной логике по вектору функции и вектору поляризации.

В 2012 году Маркеловым Н. К. была получена нижняя оценка функции Шеннона для функции трехзначной логики в классе
поляризованных полиномов \cite{mn12}:
$$
L_3(n) \geqslant \left[\frac{3}{4}3^n\right].
$$

\makeatletter
\renewcommand*{\@biblabel}[1]{\hfill#1.}
\makeatother

\newpage

\begin{thebibliography}{0}
\bibitem{sb90} Sasao T., Besslich P. On the complexity of mod-2 sum PLA’s  // IEEE Trans.on Comput. 39. N 2. 1990. P.~262--266. 
\bibitem{sv93} Супрун~В.\,П. Сложность булевых функций в классе канонических поляризованных полиномов // Дискретная математика. 5.
    \textnumero 2. 1993. С. 111--115. 
\bibitem{pn95} Перязев Н.\,А. Сложность булевых функций в классе полиномиальных поляризованных~форм // Алгебра и логика. 34.
    \textnumero 3. 1995. С. 323--326. 
\bibitem{ss02} Селезнева С.\,H. О сложности представления функций многозначных логик поляризованными полиномами. Дискретная
    математика. 14. \textnumero 2. 2002. С.~48--53.
\bibitem{kk05} Кириченко~К.\,Д. Верхняя оценка сложности полиномиальных нормальных форм булевых функций 
    // Дискретная математика. 17. \textnumero 3. 2005. С. 80--88.
\bibitem{sd08} Селезнева С.\,Н. Дайняк А.\,Б. О сложности обобщенных полиномов k\nobreakdash-значных функций // Вестник Московского
    университета. Серия 15. Вычислительная математика и кибернетика. \textnumero 3. 2008. С. 34--39.
\bibitem{sm09} Селезнева С.\,H. Маркелов Н.\,К. Быстрый алгоритм построения векторов коэффициэнтов поляризованных полиномов
    k-значных функций // Ученые записки Казанского университета. Серия Физико-математические науки. 2009. 151.
    \textnumero 2 С.~147-151.
\bibitem{mn12} Маркелов Н.\,К. Нижняя оценка сложности функций трехзначной логики в классе поляризованных полиномов // Вестник
    Московского университета. Серия 15. Вычислительная математика и кибернетика. \textnumero 3. 2012. С. 40--45.
\bibitem{ue04} Угрюмов~Е.\,П. Цифровая схемотехника. СПб.: БХВ-Петербург, 2004. 

\end{thebibliography}

\end{document}
