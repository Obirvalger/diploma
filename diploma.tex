\documentclass[a4paper, 14pt]{extarticle}
\usepackage[utf8]{inputenc}
\usepackage[russian]{babel}
\usepackage{longtable, moreverb}
\usepackage{ amssymb, latexsym, amsmath, amsthm}
\newtheorem{myth}{Теорема}
% \newtheorem*{myth}{Теорема}
% \newcommand*{\hm}[1]{#1\nobreak\discretionary{}%
    % {\hbox{$\mathsurround=0pt #1$}}{}}
\sloppy

\xdef\LastDeclaredEncoding{T2A}
% Переносы математики.
\begingroup
\catcode`\+\active\gdef+{\mathchar8235\nobreak\discretionary{}{\usefont{OT1}{cmr}{m}{n}\char43}{}}
\catcode`\-\active\gdef-{\mathchar8704\nobreak\discretionary{}{\usefont{OMS}{cmsy}{m}{n}\char0}{}}
\catcode`\=\active\gdef={\mathchar12349\nobreak\discretionary{}{\usefont{OT1}{cmr}{m}{n}\char61}{}}
\catcode`\<\active\gdef<{\mathchar"313C\nobreak\discretionary{}{\usefont{OML}{cmm}{m}{n}\char60}{}}
\catcode`\>\active\gdef>{\mathchar"313E\nobreak\discretionary{}{\usefont{OML}{cmm}{m}{n}\char62}{}}
\endgroup
\def\times{\mathchar8706\nobreak\discretionary{}{\usefont{OMS}{cmsy}{m}{n}\char2}{}}
\def\subset{\mathchar"321A\nobreak\discretionary{}{\usefont{OMS}{cmsy}{m}{n}\char26}{}}
%\supset,\subseteq,\notin
\def\supset{\mathchar"321B\nobreak\discretionary{}{\usefont{OMS}{cmsy}{m}{n}\char27}{}}
\def\subseteq{\mathchar"3212\nobreak\discretionary{}{\usefont{OMS}{cmsy}{m}{n}\char18}{}}
\def\neq{\not=\nobreak\discretionary{}{\usefont{OMS}{cmsy}{m}{n}\char54\usefont{OT1}{cmr}{m}{n}\char61}{}}
\def\sim{\mathchar"3218\nobreak\discretionary{}{\usefont{OMS}{cmsy}{m}{n}\char24}{}}
\def\in{\mathchar"3232\nobreak\discretionary{}{\usefont{OMS}{cmsy}{m}{n}\char50}{}}
\def\to{\mathchar"3221\nobreak\discretionary{}{\usefont{OMS}{cmsy}{m}{n}\char33}{}}
\def\?#1{#1\nobreak\discretionary{}{\hbox{$\mathsurround=0pt #1$}}{}}
% Конец переносов математики.

\begin{document}
% Восстанавливаем после amsmath.
\begingroup \catcode`\"=12
\gdef\newmcodes@{\mathcode`\'39\mathcode`\*42\mathcode`\."613A%
\mathcode`\-"8000\mathcode`\/47\mathcode`\:"603A\relax}%
\endgroup
\mathcode`\=="8000 \mathcode`\+="8000 \mathcode`\-="8000
\mathcode`\<="8000 \mathcode`\>="8000

%\tableofcontents

%\newpage

\setcounter{secnumdepth}{-1}

\section{Введение}
Одним из стандартных способов задания функций k\nobreakdash-значной логики являются поляризованные полиномиальные формы (ППФ),
которые также называются обобщенными формами Рида-Мюллера, или каноническими поляризованными полиномами. В ППФ каждая переменная
имеет определенную поляризацию. Длиной полиномиальной формы называется число попарно различных слагаемых в ней. Длиной функции
$F$ в классе ППФ называется наименьшая длина среди длин всех поляризованных полиномиальных форм, реализующих $F$.
%Для функции k\nobreakdash-значной логики $F$ сложностью в классе поляризованных полиномиальных форм называется
%длина кратчайшей ППФ, реализующей~$F$.
%~ Поляризованные полиномиальные формы не 
Функция Шеннона $L^K_k(n)$ длины определяется как наибольшая длина среди всех функций $k$\nobreakdash-значной логики в классе $K$
от~$n$~переменных, если $K$ опущено, то подразумевается класс ППФ.
Практическое применение ППФ нашли при построении программируемых логических матриц (ПЛМ)~\cite{ue04, sb90}, сложность ПЛМ
напрямую зависит от длины ППФ, по которой она построена. Поэтому в ряде работ исследуется сложность ППФ различных функций.

В 1993  В.\,П.\,Супрун~\cite{sv93} получил первые оценки функции Шеннона для функций алгебры логики :
$$
L_2(n) \geqslant C_n^{[\frac{n}{2}]},
$$
$$
L_2(n) < 3 \cdot 2^{n-1}.
$$
где [$a$] обозначает целую часть $a$.

Точное значение функции Шеннона для функций алгебры логики в 1995\,г. было
найдено Н.\,А.\,Перязевым~\cite{pn95} :
$$
L_2(n) = \left[\frac{2^{n+1}}{3}\right].
$$

Функции $k$\nobreakdash-значных логик являются естественным обобщением функций алгебры логики.
Для функций $k$\nobreakdash-значной логики верхняя оценка функции Шеннона была получена в 2002\,г. С.\,Н.\,Селезневой~\cite{ss02} :
$$
L_k(n) < \frac{k(k-1)}{k(k-1)+1}k^n.
$$

При построении ПЛМ рассматривают и другие полиномиальные формы. Например класс обобщенных полиномиальных форм.
В классе обобщенных полиномиальных форм, в отличие от класса поляризованных полиномиальных форм, переменные могут иметь
различную поляризацию в разных слагаемых. В статье К.\,Д.\,Кириченко~\cite{kk05}, опубликованной в 2005\,г., получена верхняя оценка
функции Шеннона в классе обобщенных полиномиальных форм функций алгебры логики :
$$
L^{\text{О.П.}}_2(n) < \frac{2 ^ {n + 1}(\log_2n+1)}{n}.
$$

Верхняя оценка функции Шеннона в классе обобщенных полиномиальных форм функций k\nobreakdash-значной логики была получена
С.\,Н.\,Селезневой А.\,Б.\,Дайняком в 2008\,г.~\cite{sd08}:
$$
L^{\text{О.П.}}_k(n) \lesssim 2\cdot\frac{k ^ n}{n}\cdot \ln n \text{ при } n \rightarrow \infty.
$$

В 2012\,г. Н.\,К.\,Маркеловым была получена нижняя оценка функции Шеннона для функции трехзначной логики в классе
поляризованных полиномов~\cite{mn12}:
$$
L_3(n) \geqslant \left[\frac{3}{4}3^n\right].
$$

\section{Теоремы}

\begin{myth} При $n \geqslant 1 $ для периодических функций пятизначной логики $f_n = f^{\left(n\right)}_{\left(1144\right)}$,
$g_n = f^{\left(n\right)}_{\left(1441\right)}$ верны следующие равенства:
\end{myth}

$ f_{n+1} = j_0(x_{n+1})f_n + j_1(x_{n+1})g_n - j_2(x_{n+1})f_n - j_3(x_{n+1})g_n + j_4(x_{n+1})f_n
 = 4 \, f_{n} x_{n+1}^{4} + {\left(3 \, f_{n} + 2 \, g_{n}\right)} x_{n+1}^{3} + {\left(3 \, f_{n} + 3 \, g_{n}\right)} x_{n+1}^{2} + {\left(4 \, f_{n} + g_{n}\right)} x_{n+1} + f_{n}
 = 4 \, f_{n} {\left(x_{n+1} + 1\right)}^{4} + {\left(2 \, f_{n} + 2 \, g_{n}\right)} {\left(x_{n+1} + 1\right)}^{3} + {\left(3 \, f_{n} + 2 \, g_{n}\right)} {\left(x_{n+1} + 1\right)}^{2} + {\left(f_{n} + g_{n}\right)} {\left(x_{n+1} + 1\right)} + f_{n}
 = 4 \, f_{n} {\left(x_{n+1} + 2\right)}^{4} + {\left(f_{n} + 2 \, g_{n}\right)} {\left(x_{n+1} + 2\right)}^{3} + {\left(f_{n} + g_{n}\right)} {\left(x_{n+1} + 2\right)}^{2} + 3 \, g_{n} {\left(x_{n+1} + 2\right)} + 4 \, g_{n}
 = 4 \, f_{n} {\left(x_{n+1} + 3\right)}^{4} + 2 \, g_{n} {\left(x_{n+1} + 3\right)}^{3} + 2 \, f_{n} {\left(x_{n+1} + 3\right)}^{2} + 2 \, g_{n} {\left(x_{n+1} + 3\right)} + 4 \, f_{n}
 = 4 \, f_{n} {\left(x_{n+1} + 4\right)}^{4} + {\left(4 \, f_{n} + 2 \, g_{n}\right)} {\left(x_{n+1} + 4\right)}^{3} + {\left(f_{n} - g_{n}\right)} {\left(x_{n+1} + 4\right)}^{2} + 3 \, g_{n} {\left(x_{n+1} + 4\right)} + g_{n}
$

\begin{myth} При $n \geqslant 1 $ для периодических функций пятизначной логики $f_n = f^{\left(n\right)}_{\left(1144\right)}$,
$g_n = f^{\left(n\right)}_{\left(1441\right)}$ верны следующие равенства:
\end{myth}

$ g_{n+1} = j_0(x_{n+1})g_n - j_1(x_{n+1})f_n - j_2(x_{n+1})g_n + j_3(x_{n+1})f_n + j_4(x_{n+1})g_n
 = 4 \, g_{n} x_{n+1}^{4} + {\left(3 \, f_{n} + 3 \, g_{n}\right)} x_{n+1}^{3} + {\left(2 \, f_{n} + 3 \, g_{n}\right)} x_{n+1}^{2} + {\left(4 \, f_{n} + 4 \, g_{n}\right)} x_{n+1} + g_{n}
 = 4 \, g_{n} {\left(x_{n+1} + 1\right)}^{4} + {\left(3 \, f_{n} + 2 \, g_{n}\right)} {\left(x_{n+1} + 1\right)}^{3} + {\left(3 \, f_{n} + 3 \, g_{n}\right)} {\left(x_{n+1} + 1\right)}^{2} + {\left(4 \, f_{n} + g_{n}\right)} {\left(x_{n+1} + 1\right)} + g_{n}
 = 4 \, g_{n} {\left(x_{n+1} + 2\right)}^{4} + {\left(3 \, f_{n} + g_{n}\right)} {\left(x_{n+1} + 2\right)}^{3} + 4 \, {\left(f_{n} + 4 \, g_{n}\right)} {\left(x_{n+1} + 2\right)}^{2} + 2 \, f_{n} {\left(x_{n+1} + 2\right)} + f_{n}
 = 4 \, g_{n} {\left(x_{n+1} + 3\right)}^{4} + 3 \, f_{n} {\left(x_{n+1} + 3\right)}^{3} + 2 \, g_{n} {\left(x_{n+1} + 3\right)}^{2} + 3 \, f_{n} {\left(x_{n+1} + 3\right)} + 4 \, g_{n}
 = 4 \, g_{n} {\left(x_{n+1} + 4\right)}^{4} + {\left(3 \, f_{n} + 4 \, g_{n}\right)} {\left(x_{n+1} + 4\right)}^{3} + {\left(f_{n} + g_{n}\right)} {\left(x_{n+1} + 4\right)}^{2} + 2 \, f_{n} {\left(x_{n+1} + 4\right)} + 4 \, f_{n}
$


\makeatletter
\renewcommand*{\@biblabel}[1]{\hfill#1.}
\makeatother

\newpage

\begin{thebibliography}{0}
\bibitem{ue04} Угрюмов~Е.\,П. Цифровая схемотехника. СПб.: БХВ-Петербург, 2004.
\bibitem{sb90} Sasao T., Besslich P. On the complexity of mod-2 sum PLA’s  // IEEE Trans.on Comput. 39. N 2. 1990. P.~262--266. 
\bibitem{sv93} Супрун~В.\,П. Сложность булевых функций в классе канонических поляризованных полиномов // Дискретная математика. 5.
    \textnumero 2. 1993. С. 111--115. 
\bibitem{pn95} Перязев Н.\,А. Сложность булевых функций в классе полиномиальных поляризованных~форм // Алгебра и логика. 34.
    \textnumero 3. 1995. С. 323--326. 
\bibitem{ss02} Селезнева С.\,H. О сложности представления функций многозначных логик поляризованными полиномами. Дискретная
    математика. 14. \textnumero 2. 2002. С.~48--53.
\bibitem{kk05} Кириченко~К.\,Д. Верхняя оценка сложности полиномиальных нормальных форм булевых функций 
    // Дискретная математика. 17. \textnumero 3. 2005. С. 80--88.
\bibitem{sd08} Селезнева С.\,Н. Дайняк А.\,Б. О сложности обобщенных полиномов k\nobreakdash-значных функций // Вестник Московского
    университета. Серия 15. Вычислительная математика и кибернетика. \textnumero 3. 2008. С. 34--39.
\bibitem{mn12} Маркелов Н.\,К. Нижняя оценка сложности функций трехзначной логики в классе поляризованных полиномов // Вестник
    Московского университета. Серия 15. Вычислительная математика и кибернетика. \textnumero 3. 2012. С. 40--45.
\bibitem{sm09} Селезнева С.\,H. Маркелов Н.\,К. Быстрый алгоритм построения векторов коэффициэнтов поляризованных полиномов
    k-значных функций // Ученые записки Казанского университета. Серия Физико-математические науки. 2009. 151.
    \textnumero 2 С.~147-151.
 
\end{thebibliography}

\end{document}
